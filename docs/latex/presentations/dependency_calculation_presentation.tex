\documentclass{beamer}

% Theme and color theme
\usetheme{Boadilla}
\usecolortheme{default}

% Packages
\usepackage{amsmath}
\usepackage{graphicx}
\usepackage{array} % Para mejorar las tablas
\usepackage{booktabs} % Para líneas profesionales en tablas

% Information to be shown in the title page
\title{Calculating Economic Security Dependencies}
\subtitle{A Comprehensive Methodology for Trade Vulnerability Analysis}
\author{Manuel A. Hidalgo, Miguel Otero}
\institute{Real Instituto Elcano}
\date{\today}

\begin{document}

% Title page
\begin{frame}
  \titlepage
\end{frame}

% Slide 2: Data Source
\begin{frame}{USITC ITPD-E Database}
  \begin{itemize}
    \item Source: U.S. International Trade Commission (USITC)
    \vspace{0.3em}
    \item Content: International and domestic trade flows, reconciled for accuracy
    \vspace{0.3em}
    \item Coverage: 264 countries, 170 industries (agriculture, mining, energy, manufacturing, services)
    \vspace{0.3em}
    \item Timeline: 1986-2022 (Release 3, June 2025)
    \vspace{0.3em}
    \item Reliability: Administrative data only, no statistical estimates
  \end{itemize}
\end{frame}

% Slide 3: Introduction
\begin{frame}{Introduction: What is Economic Dependency?}
  \begin{itemize}
    \item The indicator measures a country's reliance on another as a supplier for specific goods
    \vspace{1em}
    \item Crucial for understanding vulnerabilities and risks in global supply chains
    \vspace{1em}
    \item The analysis covers both direct and indirect trade relationships:
    \begin{itemize}
        \item Direct: Country A imports goods directly from Country B
        \item Indirect: Country A imports from C, which sources from B
    \end{itemize}
    \vspace{0.5em}
    \item Essential for economic security policy design and strategic autonomy assessment
  \end{itemize}
\end{frame}

% Slide 4: Data Foundation
\begin{frame}{Foundation: Clean and Significant Trade Data}
  The analysis is built upon a solid data foundation:
  \begin{itemize}
    \item[1.] We begin with ITPD-E bilateral trade data for hundreds of industrial sectors
    \vspace{1em}
    \item[2.] Minor or insignificant trade flows are filtered out to ensure accuracy
    \vspace{0.5em}
    \begin{itemize}
        \item Trade links included only if they represent meaningful share of imports
        \item Threshold-based filtering to reduce statistical noise
    \end{itemize}
    \vspace{1em}
    \item[3.] This process focuses analysis on economically relevant dependencies
  \end{itemize}
\end{frame}

% Slide 5: Direct Dependency
\begin{frame}{Step 1: Calculating Direct Dependency}
  This measures the share of a country's total imports that comes directly from a single supplier.
  \vspace{1em}
  
  Formula:
  \begin{equation}
  D_{ij} = \frac{X_{ji}}{\sum_{k} X_{ki}}
  \end{equation}
  
  Where:
  \begin{itemize}
    \item $D_{ij}$ is the direct dependency of country $i$ on country $j$
    \item $X_{ji}$ is the value of goods imported by $i$ from $j$
    \item $\sum_{k} X_{ki}$ is the total value of goods imported by $i$ from all countries $k$
  \end{itemize}
  
  \vspace{0.5em}
  Interpretation: Values range from 0 (no dependency) to 1 (complete dependency)
\end{frame}

% Slide 6: Indirect Dependency
\begin{frame}{Step 2: Indirect Dependency \& Path Analysis}
  We reveal hidden relationships by tracing goods through intermediary countries.
  \vspace{0.5em}

  Path Dependency Formula:
  \begin{equation}
  \text{Dependence}(p) = \frac{f_{p_1, p_2}}{d_{p_n}} \times \prod_{i=2}^{n-1} \frac{f_{p_i, p_{i+1}}}{d_{p_i}}
  \end{equation}
  
  Where:
  \begin{itemize}
    \item $p$ represents a supply chain path from origin to destination
    \item $f_{a,b}$ is the trade flow from country $a$ to country $b$
    \item $d_i$ is the total demand of country $i$
    \item The product captures "dilution" effects through intermediaries
  \end{itemize}
  
  \vspace{0.5em}
  Key insight: Captures vulnerability transmission through complex supply networks
\end{frame}

% Slide 7: Intermediary Analysis
\begin{frame}{Step 3: The Role of Intermediaries}
  We identify critical roles each country plays within global supply chains.
  \vspace{1em}
  
  Two key metrics for every intermediary country:
  \vspace{1em}

  \begin{table}[h]
  \centering
  \begin{tabular}{l p{0.65\textwidth}}
    \toprule
    Frequency & How often the country acts as a "bridge" in global supply paths \\
    \midrule
    Strength & Total economic value of trade flowing through the country as intermediary \\
    \bottomrule
  \end{tabular}
  \end{table}

  \vspace{1em}
  This allows identification of:
  \begin{itemize}
    \item Critical hubs in the global trade network
    \item Potential bottlenecks and single points of failure
    \item Systemically important intermediary countries
  \end{itemize}
\end{frame}

% Slide 8: Final Indicator
\begin{frame}{The Final Indicator: Total Dependency}
  The total dependency score combines all layers of analysis.
  \vspace{1em}
  
  \begin{equation}
  \text{Total Dependency}_{ij} = D_{ij}^{\text{direct}} + \sum_{\text{paths } p} \text{Dependence}(p)
  \end{equation}
  
  \vspace{1em}
  This provides:
  \begin{itemize}
    \item Holistic view of economic vulnerability to specific suppliers
    \item Ranking of most critical supply paths
    \item Policy-relevant insights for strategic autonomy
    \item Scenario analysis capabilities for different disruption patterns
  \end{itemize}
  
  \vspace{1em}
  Applications: Economic security assessment, diversification strategy design, resilience policy planning
\end{frame}

% Slide 9: Policy Applications
\begin{frame}{Policy Applications for Economic Security}
  The methodology supports evidence-based policy design:
  \vspace{1em}
  
  Strategic Applications:
  \begin{itemize}
    \item Vulnerability Assessment: Identify critical dependencies before crises
    \item Diversification Strategy: Distinguish between superficial and deep diversification
    \item Resilience Planning: Prioritize sectors and relationships for intervention
    \item Scenario Analysis: Evaluate impact of geopolitical fragmentation
  \end{itemize}
  
  \vspace{1em}
  Particularly relevant for:
  \begin{itemize}
    \item EU "Open Strategic Autonomy" initiatives
    \item Critical materials and technologies assessment
    \item Supply chain resilience in strategic sectors
  \end{itemize}
\end{frame}

% Slide 10: Future Work
\begin{frame}{Future Research Directions}
  Extensions to enhance the methodology:
  \vspace{1em}
  
  Value Chain Integration:
  \begin{itemize}
    \item Explicit inclusion of global value chains in dependency calculations
    \item Analysis beyond directly traded products to critical input requirements
    \item Mapping dependencies of products that require other critical goods for production
    \item Integration of input-output relationships within supply chain analysis
  \end{itemize}
  
  \vspace{1em}
  Additional Enhancements:
  \begin{itemize}
    \item Time-varying dependency patterns and trend analysis
    \item Integration of geopolitical risk factors
    \item Sectoral vulnerability indices incorporating technological complexity
    \item Dynamic simulation of disruption scenarios
  \end{itemize}
\end{frame}

\end{document}