% CORRECCIONES CLAVE PARA LA SECCIÓN DE METODOLOGÍA

\subsection{Correcciones a las Ecuaciones}

% ========================================
% DEPENDENCIA DIRECTA (CORREGIDA)
% ========================================
\subsubsection{Dependencia Directa}

La dependencia directa de un país importador $j$ respecto a un exportador $i$ se define como la proporción de las importaciones totales de $j$ que provienen de $i$:

\begin{equation}
\text{DD}_{ij} = \frac{x_{ij}}{S_j}
\end{equation}

\noindent donde:
\begin{itemize}
    \item $x_{ij}$ representa el flujo comercial del país exportador $i$ al país importador $j$
    \item $S_j = \sum_{k} x_{kj}$ es el total de importaciones del país $j$ (suma por columna)
\end{itemize}

\textbf{Nota técnica:} En esta formulación, asumimos que la matriz $\mathbf{X}$ tiene exportadores en filas e importadores en columnas. La producción doméstica ($x_{jj}$) se excluye del denominador para enfocar el análisis en dependencias comerciales internacionales.

% ========================================
% FUERZA DEL CAMINO (CORREGIDA)
% ========================================
\subsubsection{Fuerza del Camino}

Para un camino $p = (i, k_1, k_2, \ldots, k_{\ell-1}, j)$ que conecta el exportador $i$ con el importador $j$ a través de intermediarios, la fuerza del camino se calcula como el producto de las proporciones comerciales en cada salto:

\begin{equation}
F(p) = \prod_{(a,b) \in \text{aristas}(p)} \frac{x_{a,b}}{S_b}
\end{equation}

\noindent Expandiendo para el camino completo:

\begin{equation}
F(p) = \frac{x_{i,k_1}}{S_{k_1}} \cdot \frac{x_{k_1,k_2}}{S_{k_2}} \cdots \frac{x_{k_{\ell-1},j}}{S_j}
\end{equation}

Esta formulación captura cómo una disrupción en el exportador original $i$ se transmitiría a través de la cadena hasta el importador final $j$, atenuándose en cada intermediario según la diversificación de sus proveedores.

% ========================================
% DEPENDENCIA INDIRECTA (ACLARADA)
% ========================================
\subsubsection{Dependencia Indirecta}

La dependencia indirecta captura todas las rutas con al menos un intermediario ($\ell \geq 2$):

\begin{equation}
\text{DI}_{ij} = \sum_{p \in P_{ij}^{\ell \geq 2}} F(p)
\end{equation}

\noindent donde $P_{ij}^{\ell \geq 2}$ es el conjunto de todos los caminos simples (sin repetir nodos) desde $i$ hasta $j$ con longitud mínima 2.

\textbf{Algoritmo implementado:} El código utiliza una enumeración exhaustiva mediante \texttt{combinations} para longitudes 3 y superiores, mientras que para longitud 2 emplea una optimización vectorizada:

\begin{equation}
\text{DI}_{ij}^{(\ell=2)} = \sum_{k \neq i,j} \frac{x_{i,k}}{S_k} \cdot \frac{x_{k,j}}{S_j}
\end{equation}

% ========================================
% MATRIZ DE TRANSICIÓN (NUEVA)
% ========================================
\subsubsection{Matriz de Transición Normalizada}

Para facilitar cálculos, definimos la matriz de transición $\mathbf{T}$ donde cada elemento representa la proporción del comercio:

\begin{equation}
T_{ij} = \frac{x_{ij}}{S_j} = \frac{x_{ij}}{\sum_k x_{kj}}
\end{equation}

Esta matriz permite expresar la fuerza de cualquier camino como:

\begin{equation}
F(p) = \prod_{(a,b) \in p} T_{ab}
\end{equation}

% ========================================
% CONVERGENCIA (ACLARADA)
% ========================================
\subsubsection{Criterios de Convergencia}

El algoritmo aplica tres criterios para limitar la búsqueda de caminos:

\begin{enumerate}
    \item \textbf{Longitud máxima:} $L_{\max} = 3$ (configurable)
    \item \textbf{Umbral de contribución:} Se descartan caminos con $F(p) < \theta_{\text{path}} = 10^{-3}$
    \item \textbf{Convergencia incremental:} Se detiene la iteración cuando:
    \begin{equation}
    \left| \text{DT}_{ij}^{(\ell)} - \text{DT}_{ij}^{(\ell-1)} \right| < \varepsilon = 0.01
    \end{equation}
\end{enumerate}

Estos criterios garantizan eficiencia computacional sin sacrificar precisión significativa en el cálculo de dependencias relevantes.
