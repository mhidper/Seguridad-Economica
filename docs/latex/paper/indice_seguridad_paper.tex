
% ============================================================
% Índice de Seguridad — Template de paper (versión inicial)
% Compila con: pdflatex (dos pasadas). Bibliografía opcional.
% ============================================================

\documentclass[11pt,a4paper]{article}

% ---------- Paquetes mínimos ----------
\usepackage[utf8]{inputenc}
\usepackage[T1]{fontenc}
\usepackage[spanish]{babel}
\usepackage{geometry}
\geometry{margin=2.5cm}
\usepackage{amsmath, amssymb}
\usepackage{booktabs}
\usepackage{graphicx}
\usepackage{hyperref}
\usepackage{enumitem}
\usepackage{siunitx}
\sisetup{output-decimal-marker={,}}
\usepackage{csquotes}

% ---------- Metadatos ----------
\title{Índice de Seguridad: un indicador de dependencia comercial directa e indirecta}
\author{%
  Equipo Índice de Seguridad\\
  \small \texttt{contacto@indice-seguridad.org}
}
\date{\today}

% ---------- Macros y notación ----------
% Comentarios editoriales: busca %CHATGPT: en el código
% Notación estándar
\newcommand{\X}{X}  % matriz de flujos bilaterales (exportador u, importador v)
\newcommand{\T}{T}  % matriz de transición por columnas
\newcommand{\DD}{\mathrm{DD}} % dependencia directa
\newcommand{\DI}{\mathrm{DI}} % dependencia indirecta
\newcommand{\DT}{\mathrm{DT}} % dependencia total
\newcommand{\hidden}{\mathrm{Hidden}} % hidden dependence
\newcommand{\Lmax}{L_{\max}} % longitud máxima de camino
\newcommand{\epsedge}{\epsilon_{\text{edge}}} % umbral de arista
\newcommand{\epscontrib}{\epsilon_{\text{contrib}}} % umbral de contribución
\newcommand{\topm}{\texttt{top\_m}} % número máximo de vecinos relevantes

\begin{document}
\maketitle

\begin{abstract}
\noindent
%CHATGPT: En 150--200 palabras, explica qué mide el indicador (exposición a disrupciones de suministro a través de rutas directas e indirectas), en qué datos se basa (matrices bilaterales por industria) y un resultado principal. Incluye 1--2 cifras headline si ya están listas.
\end{abstract}

\section{Introducción y motivación}
%CHATGPT: Contextualiza riesgos de disrupción y concentración. Intuición de por qué las dependencias indirectas importan (encadenamientos vía terceros). Resume contribuciones en 3 viñetas.
\begin{itemize}[leftmargin=*,nosep]
  \item Proponemos un indicador \emph{trazable} de dependencia total (\DT) desagregado por país y por industria.
  \item Distinguimos \DD\ (bilateral) de \DI\ (rutas con intermediarios) y caracterizamos ``\emph{hidden dependence}'' cuando \DT/\DD\,$\gg$\,1.
  \item Ofrecemos salidas operativas (rutas críticas, intermediarios) útiles para vigilancia de riesgo.
\end{itemize}

\section{Relación con la literatura}
%CHATGPT: 5--8 párrafos. Redes de comercio, centralidad/intermediación, medidas de dependencia en IO/valor añadido, papers sobre riesgo de disrupciones. Evita tecnicismos aquí; reserva detalles para Metodología.

\section{Datos}
%CHATGPT: Fuente, cobertura temporal y sectorial, construcción de matrices exportador×importador por industria, filtros previos y umbrales. Dejar claro que los cálculos se hacen por industria (normalmente) y luego se pueden agregar.
\begin{itemize}[leftmargin=*,nosep]
  \item Universo de países: \textit{[completar]}; industrias: \textit{[completar]}.
  \item Matrices \(\X^{(\ell)}\) por industria \(\ell\): entrada \(\X_{uv}^{(\ell)}\) es el flujo de \(u\to v\).
  \item Limpieza: corte relativo por columna (importador) y eliminación de columnas/filas nulas tras el corte.
\end{itemize}

\section{Metodología del indicador}
%CHATGPT: Sección normativa y ``limpia''. Mantener esta parte independiente de implementación.
\subsection{Dependencia directa}
Para un par país destino \(i\) y origen \(j\), definimos la dependencia directa como
\begin{equation}
  \DD_{ij} \;=\; \frac{\X_{ji}}{\sum_{k} \X_{ki}},
  \label{eq:dd}
\end{equation}
donde el denominador suma todas las importaciones de \(i\).

\subsection{Matriz de transición por columnas}
Definimos la matriz de transición \(\T\) con normalización por columnas:
\begin{equation}
  \T_{uv} \;=\; \frac{\X_{uv}}{\sum_{a}\X_{av}}, \qquad
  \text{con la convención } \sum_{a}\X_{av}=0 \Rightarrow \T_{uv}=0.
  \label{eq:T}
\end{equation}
%CHATGPT: \T\ se interpreta como la proporción del suministro total de \(v\) que proviene de \(u\).

\subsection{Dependencia indirecta (rutas)}
Sea \(\mathcal{P}_{j\leadsto i}\) el conjunto de rutas simples (sin repetir nodos) desde \(j\) hasta \(i\) con longitudes \(2,\dots,\Lmax\).
La contribución de una ruta \(p=(j \to \cdots \to i)\) es el producto de transiciones:
\begin{equation}
  \mathrm{contrib}(p) \;=\; \prod_{(u\to v)\in p} \T_{uv}.
  \label{eq:path}
\end{equation}
La dependencia indirecta de \(i\) respecto a \(j\) es
\begin{equation}
  \DI_{ij} \;=\; \sum_{p\in \mathcal{P}_{j\leadsto i}} \mathrm{contrib}(p).
  \label{eq:di}
\end{equation}
%CHATGPT: Operamos con poda: ignoramos aristas con \(\T_{uv}<\epsedge\); cortamos rutas si la contribución acumulada cae por debajo de \(\epscontrib\); y limitamos la exploración a \topm\ vecinos por nodo.

\subsection{Dependencia total y ``hidden dependence''}
\begin{equation}
  \DT_{ij} \;=\; \DD_{ij} + \DI_{ij}.
\end{equation}
Diremos que existe \emph{hidden dependence} cuando \(\DT_{ij}/\DD_{ij} \gg 1\) para un umbral a definir (p.~ej., \(>2\)).

\subsection{Intermediarios críticos}
Definimos una puntuación para un país \(m\) como intermediario en rutas hacia \(i\) que parten de \(j\)
combinando (i) la \textit{frecuencia} con que \(m\) aparece en rutas de mayor contribución y (ii) la \textit{fuerza} acumulada de dichas rutas. La métrica compuesta puede escribirse como
\begin{equation}
  \mathrm{IC}_{m;ij} \;=\; \alpha\cdot \mathrm{freq}_{m;ij} \;+\; (1-\alpha)\cdot \mathrm{strength}_{m;ij},\quad \alpha\in[0,1].
\end{equation}
%CHATGPT: En los ejercicios actuales, \(\alpha\approx0{,}4\), pero reportar sensibilidad en la sección de robustez.

\section{Implementación computacional (reproducibilidad)}
%CHATGPT: Resume el algoritmo y parámetros operativos. Detalles extensos al apéndice.
\begin{itemize}[leftmargin=*,nosep]
  \item Precalcular \(\T\) por industria; filtrar aristas con \(\T_{uv}<\epsedge\).
  \item Para cada par \((j,i)\), explorar rutas con DFS con poda (\Lmax, \epscontrib, \topm).
  \item Paralelizar por pares \((j,i)\). Almacenar: \DD, \DI, \DT, rutas críticas y métricas de intermediación.
\end{itemize}

\section{Resultados}
%CHATGPT: Figuras/tablas sugeridas:
\begin{itemize}[leftmargin=*,nosep]
  \item Distribución de \DD, \DI, \DT\ por país destino e industria.
  \item Top 10 dependencias totales por industria y sus rutas críticas asociadas.
  \item Ranking de intermediarios críticos.
\end{itemize}

\section{Robustez y sensibilidad}
%CHATGPT: Sensibilidad a \Lmax, \epsedge, \epscontrib, \topm\ y al umbral de limpieza de datos. Incluye tablas con variación porcentual en \DT.

\section{Discusión}
%CHATGPT: Interpretación económica y de política pública. Casos ilustrativos (rutas con concentración elevada, países-puente).

\section{Conclusiones}
%CHATGPT: Qué aporta el indicador, qué limita, y líneas futuras (actualización temporal, integración con shocks reales, etc.).

\appendix
\section*{Apéndices}

\subsection*{A. Tabla de notación}
\begin{table}[h!]
\centering
\begin{tabular}{ll}
\toprule
Símbolo & Descripción \\\midrule
\(\X_{uv}\) & Flujo de comercio desde \(u\) (origen) hacia \(v\) (destino) \\
\(\T_{uv}\) & Proporción del suministro de \(v\) que proviene de \(u\) (normalización por columnas) \\
\(\DD_{ij}\) & Dependencia directa de \(i\) respecto a \(j\) \\
\(\DI_{ij}\) & Dependencia indirecta (suma de rutas) \\
\(\DT_{ij}\) & Dependencia total \\
\(\Lmax\) & Longitud máxima de las rutas consideradas \\
\(\epsedge\) & Umbral mínimo para conservar una arista en \(\T\) \\
\(\epscontrib\) & Umbral mínimo de contribución de una ruta durante la exploración \\
\(\topm\) & Máximo de vecinos por nodo en la exploración \\
\bottomrule
\end{tabular}
\end{table}

\subsection*{B. Pseudocódigo (DFS con poda)}
\begin{verbatim}
Input: T (matriz de transición), i (destino), j (origen),
       Lmax, eps_edge, eps_contrib, top_m

function IndirectDependence(i, j):
    total = 0
    stack = [(j, 0, 1.0)]  # (nodo actual, longitud, prod acumulado)
    while stack not empty:
        (u, L, prod) = stack.pop()
        if L >= 1 and u == i:
            total += prod
        if L == Lmax: continue
        # Seleccionar vecinos relevantes (v) con T[u,v] >= eps_edge
        for v in top_neighbors(u, top_m):
            w = T[u,v]
            if w < eps_edge: continue
            prod2 = prod * w
            if prod2 < eps_contrib: continue
            if v not in ruta_actual:  # evitar ciclos
                push (v, L+1, prod2)
    return total
\end{verbatim}

%CHATGPT: Sustituir \texttt{top\_neighbors} por el criterio real (p.ej., mayores \(\T_{uv}\)).

% ---------- Bibliografía (opcional) ----------
% \bibliographystyle{apalike}
% \bibliography{referencias}

\end{document}
